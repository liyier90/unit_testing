%%%%%%%%%%%%%%%%%%%%%%%%%%%%%%%%%%%%%%%%%%%%%%%%%%%%%%%%%%%
% code listings 
%%%%%%%%%%%%%%%%%%%%%%%%%%%%%%%%%%%%%%%%%%%%%%%%%%%%%%%%%%%
\defverbatim[colored]\coveragecode{%
\begin{minted}[fontsize=\scriptsize]{python}
def is_fizzbuzz(num: int) -> bool:
    if num % 3 and num % 5:
        return True
    return some_var

def test_fizzbuzz():
    result = is_fizzbuzz(3)
    assert result
\end{minted}
}

\defverbatim[colored]\coveragecodetwo{%
\begin{minted}[fontsize=\scriptsize]{python}
def is_fizzbuzz(num: int) -> bool:
    return True if num % 3 and num % 5 else some_var

def test_fizzbuzz():
    result = is_fizzbuzz(3)
    assert result
\end{minted}
}

\defverbatim[colored]\coveragecodebranch{%
\begin{minted}[fontsize=\scriptsize]{python}
def generate_number(
    cond_1: bool = True,
    cond_2: bool = True,
    cond_3: bool = True,
) -> int:
    if cond_1:
        num = 1
    if cond_2:
        num = 2
    if cond_3:
        num = 3
    return num
\end{minted}
}

\defverbatim[colored]\aaapattern{%
\begin{minted}[fontsize=\scriptsize]{python}
# A cohesive set of tests, optional
class TestCalculator:
    # Name of the unit test
    def test_sum_of_two_numbers(self):
        # Arrange
        first = 10
        second = 20
        calculator = Calculator()

        # Act
        result = calculator.sum(first, second)

        # Assert
        assert result == 30
\end{minted}
}

\defverbatim[colored]\avoidifpkd{%
\begin{minted}[fontsize=\scriptsize]{python}
def test_node_with_python_updates(self, req_file):
    with TestCase.assertLogs("...logger") as cap:
        assert check_requirements(
            NODE, req_file
        ) == 2
        for i, rec in enumerate(cap.records):
            idx = int(i / 2)
            if i == 4:
                assert "2 packages updated" in rec.getMessage()
            elif i % 2 == 0:
                assert f"{PY_PKGS[idx]} not found" in rec.getMessage()
            else:
                assert f"pip install {PY_PKGS[idx]}" in rec.getMessage()
\end{minted}
}

\defverbatim[colored]\rigidname{%
\begin{minted}[fontsize=\scriptsize]{python}
def test_is_delivery_valid_invalid_date_returns_false():
    sut = DeliveryService()
    past_date = datetime.today() - timedelta(days=1)
    delivery = Delivery(date=past_date)

    is_valid = sut.is_delivery_valid(delivery)

    assert not is_valid
\end{minted}
}

\defverbatim[colored]\verbosename{%
\begin{minted}[fontsize=\scriptsize]{python}
def test_delivery_with_past_date_should_be_considered_invalid():
    ...
\end{minted}
}

\defverbatim[colored]\concisename{%
\begin{minted}[fontsize=\scriptsize]{python}
def test_delivery_with_a_past_date_is_invalid():
    ...
\end{minted}
}

\defverbatim[colored]\additionalbehavior{%
\begin{minted}[fontsize=\scriptsize]{python}
def test_delivery_for_today_is_invalid():
    ...

def test_delivery_for_tomorrow_is_invalid():
    ...

def test_the_soonest_delivery_date_is_two_days_from_now():
    ...
\end{minted}
}

\defverbatim[colored]\parametrizeall{%
\begin{minted}[fontsize=\scriptsize]{python}
@pytest.mark.parametrize(
    "days_from_now,expected",
    [(-1, False), (0, False), (1, False), (2, True)],
)
def test_can_detect_an_invalid_delivery_date(
    days_from_now, expected
):
    sut = DeliveryService()
    delivery_date = datetime.today() + timedelta(days=days_from_now)
    delivery = Delivery(date=delivery_date)

    is_valid = sut.is_delivery_valid(delivery)

    assert is_valid == expected

\end{minted}
}

\defverbatim[colored]\parametrizesome{%
\begin{minted}[fontsize=\scriptsize]{python}
@pytest.mark.parametrize("days_from_now", [-1, 0, 1])
def test_detects_an_invalid_delivery_date(days_from_now):
    ...

    assert not is_valid


def test_the_soonest_delivery_date_is_two_days_from_now():
    ...

    assert is_valid
\end{minted}
}

\defverbatim[colored]\assertionlibrary{%
\begin{minted}[fontsize=\scriptsize]{python}
def test_sum_of_two_numbers():
    ...

    assert result == 30

def test_sum_of_two_numbers():
    ...

    assert_that(result).is_equal_to(30)
\end{minted}
}

\defverbatim[colored]\chailibrary{%
\begin{minted}[fontsize=\scriptsize]{javascript}
describe("Calculator", () => {
    it("computes the sum of two number", () => {
        const calculator = new Calculator();

        const result calculator.sum(10, 20);

        expect(result).to.be.equal(30);
    });
});
\end{minted}
}

\defverbatim[colored]\rendererimpl{%
\begin{minted}[fontsize=\scriptsize]{python}
@dataclass
class Message:
    header: str
    body: str
    footer: str

class IRenderer(ABC):
    @abstractmethod
    def render(self, message: "Message") -> str:
        """Renders the provided message."""

class MessageRenderer(IRenderer):
    def __init__(self) -> None:
        self.sub_renderers = [HeaderRenderer(), BodyRenderer(), FooterRenderer()]

    def render(self, message: "Message") -> str:
        return "".join(renderer.render(message) for renderer in self.sub_renderers)

class HeaderRenderer(IRenderer):
    def render(self, message: "Message") -> str:
        return f"<head>{message.header}</head>"

...
\end{minted}
}

\defverbatim[colored]\renderertestone{%
\begin{minted}[fontsize=\scriptsize]{python}
def test_message_renderer_uses_correct_sub_renderers():
    sut = MessageRenderer()

    sub_renderers = sut.sub_renderers

    assert len(sub_renderers) == 3
    assert isinstance(sub_renderers[0], HeaderRenderer)
    assert isinstance(sub_renderers[1], BodyRenderer)
    assert isinstance(sub_renderers[2], FooterRenderer)
\end{minted}
}

\defverbatim[colored]\renderertesttwo{%
\begin{minted}[fontsize=\scriptsize]{python}
def test_message_renderer_renders_message():
    sut = MessageRenderer()
    message = Message("h", "b", "f")

    html = sut.render(message)

    assert html == "<head>h</head><body>b</body><foot>f</foot>"
\end{minted}
}